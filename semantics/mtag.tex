\documentclass{article}
\usepackage[utf8]{inputenc}
\usepackage{amsmath,amsthm,amssymb,bbm,amsfonts,syntax}
\usepackage[ligature, inference]{semantic}
\usepackage[T1]{fontenc}
\usepackage{mathpartir}
\textwidth 7.0 truein
\oddsidemargin -0.25in %left-hand edge
\evensidemargin -0.5 truein %right-hand edge
\topmargin -0.85in %top of paper to top of head, pulls whole unit
\textheight 9.5in
\setlength\parindent{0pt}
\setlength{\parskip}{.5em}

\begin{document}
\newcommand{\mat}{\mathsf{mat}n_1\textnormal{x}n_2}
\newcommand{\env}[1]{\langle#1,\sigma\rangle}

\mathlig{->}{\rightarrow}
\mathlig{|-}{\vdash}
\mathlig{=>}{\Rightarrow}
\mathligson

\title{Matrix Tagging Syntax and Semantics}
\author{Dietrich Geisler and Irene Yoon}

\section{Syntax}

$\begin{array}{rcl}
T &\in& \text{tags} \\
x &\in& \text{variables} \\
n &\in& \mathbb{N} \\
s &\in& \mathbb{R}\\
\mathsf{mat}\;m&::=&\{[[s_{11},s_{12},\dots],[s_{21},s_{22},\dots],\dots]\}\\
\mathsf{vec}\;v&::=&\{[s_1,s_2,\dots]\}\\
\tau&::=&\mathsf{unit}\;|\;\mathsf{scalar}\;|\; \mathsf{mat}_{n_1{\times}n_2} \;|\;\mathsf{vec}_n\;|\;T\;|\;\tau_1->\tau_2\\
c&::=&n\;|\;s\;|\;m \;|\; ()\\
t&::=&\mathsf{tag}\;\tau_1\;\mathsf{is}\;\tau_2\;|\;t_1;t_2\\
e&::=&\dots\;|\;c\;|\;\tau\;x:=e\;|\;\tau\;x:=(\tau)\;m\;|\;e_1+e_2\;|\;e_1*e_2\;|\;e_1\;\mathsf{.*}\;e_2\;|\\
&&e_1\textnormal{|>}e_2\;|\;\mathsf{comp}\;e_1\;e_2
\end{array}$

\subsection{Program Structure}
$t;e\;\textnormal{EOF}$\\

\section{Subtype Ordering}

% Including some English-language context for rules is important.
We define a judgment $\tau_1 \leq \tau_2$ for subtyping.
Subtyping is reflexive and transitive, as usual:
% Adrian recommends using mathpartir, which will make you less sad in the long
% run...
\begin{mathpar}
\inferrule
    { }
    {\tau\leq\tau}

\inferrule
    {\tau_1 \leq \tau_2 \and \tau_2 \leq \tau_3}
    {\tau_1 \leq \tau_3}
\end{mathpar}

This rule refers to some context $\Delta$:
%
\begin{mathpar}
\inferrule
    {\Delta(\tau_1)=\tau_2}
    {\tau_1\leq\tau_2}
\end{mathpar}

Here are some rules for subtyping on base types:
%
\begin{mathpar}
\inferrule
    { }
    {\tau\leq\mathsf{unit}}

\inferrule
    { }
    {\mathsf{int}\leq\mathsf{scalar}}

\inferrule
    { }
    {\mathsf{vec}n\leq\mathsf{mat}1\textnormal{x}n}
\end{mathpar}

This rule lets us treat transformation matrices, whose type is always given by $\tau_1 \rightarrow \tau_2$ where $\tau_1$ and $\tau_2$ are vectors, as matrices:
%
\begin{mathpar}
\inferrule
    {\tau_1\leq\mathsf{vec}n_1\qquad\tau_2\leq\mathsf{vec}n_2}
    {\tau_1->\tau_2\leq\mat}
\end{mathpar}

\section{Static Semantics}

\subsection{Tag Declarations}

\[  % Declare Tag
\inference[]{\tau_1\in T\qquad\tau_2\in T\cup v}{\Delta,\mathsf{tag}\;\tau_1\;\mathsf{is}\;\tau_2|-\Delta(\tau_1)=\tau_2}
\]

\[
% Tag sequencing
\inference[]{\Delta|-t_1->\Delta'\qquad \Delta|-t_2->\Delta''}{\Delta|-t_1;t_2->\Delta''}
\qquad\qquad
% Tag/expression sequencing
\inference[]{\Delta|-t->\Delta'\qquad\Gamma,\Delta'|-e:\tau}{\Gamma,\Delta|-t;e:\tau}
\]

\subsection{Constants and Variable Declarations}

\[  % Numeric Declaration
\Gamma,\Delta|-n:\mathsf{Int}
\qquad
\Gamma,\Delta|-s:\mathsf{Scalar}
\]

\[  % Matrix/Vector Declaration
\Gamma,\Delta|-[[s_{11},\dots,s_{1n_2}],\dots,[s_{n_11},\dots,s_{n_1n_2}]]:\mathsf{Mat}n_1\mathsf{x}n_2
\qquad
\Gamma,\Delta|-[s_1,\dots,s_n]:\mathsf{Vec}n
\]

\[  % Variable Declarations
\inference[]{\Gamma,\Delta|-e:\tau'\qquad\tau'\leq\tau}{\Gamma,\Delta,\tau\;x:=e|-\Gamma(x)=\tau}
\qquad\qquad
% Special case for tag casting
\inference[]{\Gamma,\Delta|-e:\tau'\qquad\tau\leq\tau'\qquad\tau\leq\mat}
{\Gamma,\Delta,\tau\;x:=(\tau)\;m|-\Gamma(x)=\tau}
\]

\subsection{Symmetric Binary Operations}

All operators on Scalars and Ints work as one might expect (recall that $\mathsf{Int}\leq\mathsf{Scalar}$).

We assume symmetry for typing for all following binary operations

\[  % Addition
\inference[]{\Gamma,\Delta|-e_1:\tau\qquad\Gamma,\Delta|-e_2:\tau'\qquad\tau'\leq\tau}
{\Gamma,\Delta|-e_1+e_2:\tau}
\qquad\qquad
\]

\[
% Scalar multiplication
\inference[]{\Gamma,\Delta|-e_1:\tau\qquad\Gamma,\Delta|-e_2:\tau'\qquad\tau\leq\mat\qquad\tau'\leq\mathsf{Scalar}}
{\Gamma,\Delta|-e_1*e_2:\tau}
\]

\[ % Component Multiplication
\inference[]{\Gamma,\Delta|-e_1:\tau\qquad\Gamma,\Delta|-e_2:\tau'\qquad\tau'\leq\tau\qquad\tau\leq\mathsf{vec}n}
{\Gamma,\Delta|-e_1\;\mathsf{.*}\;e_2:\mathsf{vec}n}
\]

\subsection{Asymmetric Binary Operations}

\[  % Bind
\inference[]{\Gamma,\Delta|-e_1:\tau_1\qquad\Gamma,\Delta|-e_2:\tau_1'->\tau_2\qquad\tau_1\leq\tau_1'}
{\Gamma,\Delta|-e_1\textnormal{|>}e_2:\tau_2}
\qquad\qquad
\]

\[  % Composition
\inference[]{\Gamma,\Delta|-e_1:\tau_1->\tau_2\qquad\Gamma,\Delta|-e_2:\tau_2'->\tau_3\qquad\tau_2\leq\tau_2'}
{\Gamma,\Delta|-\mathsf{comp}\;e_1\;e_2:\tau_1->\tau_3}
\qquad\qquad
\]

\section{Dynamic Semantics}

\[
% Tag declarations are thrown out at runtime
\inference[]{}{t->\sigma}
\]

\subsection{Variable Declarations}
\[
% Variable Declarations
\inference[]{}{\env{\tau\;x:=c}->\langle\mathsf{skip},\sigma[c/x]\rangle}
\qquad
% Declaration Reduction
\inference[]{\env{e}->\env{e'}}{\env{\tau\;x:=e}->\env{\tau\;x:=e'}}
\qquad
\inference[]{}{\env{\tau\;x:=(\tau)\;m}->\langle\mathsf{skip},\sigma[m/x]\rangle}
\]

\subsection{Mathematical Operations}

$\odot\in\{+,*,\mathsf{.*},\textnormal{|>}\}$

\[
% Left operation reduction
\inference[]{\env{e_1}->\env{e_1'}}{\env{e_1\odot e_2}->\env{e_1'\odot e_2}}
\qquad\qquad
% Right operation reduction
\inference[]{\env{e_2}->\env{e_2'}}{\env{c\odot e_2}->\env{c\odot e_2'}}
\]

\[
% Addition
\inference[]{}{\env{(m_1+m_2)}->\env{m_1+m_2}}
\qquad\qquad
% Scalar Multiplication
\inference[]{}{\env{(s*m)}->\env{s*m}}
\]

\[
% Component-wise multiplication
\inference[]{v_1=[v_{11},\dots,v_{1n}]\qquad v_2=[v_{21},\dots,v_{2n}]}
{\env{v_1\mathsf{.*}v_2}->\env{[v_{11}*v_{21},\dots,v_{1n}*v_{2n}]}}
\qquad\qquad
% Bind
\inference[]{}
{\env{v\textnormal{|>}m->v*m}}
\]

\[
% Left composition reduction
\inference[]{\env{e_1}->\env{e_1'}}{\env{\mathsf{comp}\;e_1\;e_2}->\env{\mathsf{comp}\;e_1'\;e_2}}
\qquad\qquad
% Right composition reduction
\inference[]{\env{e_2}->\env{e_2'}}{\env{\mathsf{comp}\;c\;e_2}->\env{\mathsf{comp}\;c\;e_2'}}
\]

\[
% Composition
\inference[]{}{\env{\mathsf{comp}\;m_1\;m_2}->\env{m_1*m_2}}
\]
\end{document}
